\documentclass{article}

\usepackage{tabularx}
\usepackage{booktabs}

\title{Problem Statement and Goals}

\author{\authname}

\date{}

%% Comments

\usepackage{color}

\newif\ifcomments\commentstrue %displays comments
%\newif\ifcomments\commentsfalse %so that comments do not display

\ifcomments
\newcommand{\authornote}[3]{\textcolor{#1}{[#3 ---#2]}}
\newcommand{\todo}[1]{\textcolor{red}{[TODO: #1]}}
\else
\newcommand{\authornote}[3]{}
\newcommand{\todo}[1]{}
\fi

\newcommand{\wss}[1]{\authornote{blue}{SS}{#1}} 
\newcommand{\plt}[1]{\authornote{magenta}{TPLT}{#1}} %For explanation of the template
\newcommand{\an}[1]{\authornote{cyan}{Author}{#1}}

%% Common Parts

\newcommand{\progname}{ProgName} % PUT YOUR PROGRAM NAME HERE
\newcommand{\authname}{Team \#, Team Name
\\ Student 1 name
\\ Student 2 name
\\ Student 3 name
\\ Student 4 name} % AUTHOR NAMES                  

\usepackage{hyperref}
    \hypersetup{colorlinks=true, linkcolor=blue, citecolor=blue, filecolor=blue,
                urlcolor=blue, unicode=false}
    \urlstyle{same}
                                

%%% Comments

\usepackage{color}

\newif\ifcomments\commentstrue %displays comments
%\newif\ifcomments\commentsfalse %so that comments do not display

\ifcomments
\newcommand{\authornote}[3]{\textcolor{#1}{[#3 ---#2]}}
\newcommand{\todo}[1]{\textcolor{red}{[TODO: #1]}}
\else
\newcommand{\authornote}[3]{}
\newcommand{\todo}[1]{}
\fi

\newcommand{\wss}[1]{\authornote{blue}{SS}{#1}} 
\newcommand{\plt}[1]{\authornote{magenta}{TPLT}{#1}} %For explanation of the template
\newcommand{\an}[1]{\authornote{cyan}{Author}{#1}}

%%% Common Parts

\newcommand{\progname}{ProgName} % PUT YOUR PROGRAM NAME HERE
\newcommand{\authname}{Team \#, Team Name
\\ Student 1 name
\\ Student 2 name
\\ Student 3 name
\\ Student 4 name} % AUTHOR NAMES                  

\usepackage{hyperref}
    \hypersetup{colorlinks=true, linkcolor=blue, citecolor=blue, filecolor=blue,
                urlcolor=blue, unicode=false}
    \urlstyle{same}
                                


\begin{document}
\setlength{\parindent}{0pt}

\maketitle

\begin{table}[hp]
\caption{Revision History} \label{TblRevisionHistory}
\begin{tabularx}{\textwidth}{llX}
\toprule
\textbf{Date} & \textbf{Developer(s)} & \textbf{Change}\\
\midrule
Jan.22, 2025 & Yuanqi Xue & First Draft\\
Jan.22, 2025 & Yuanqi Xue & Updated Based on Feedback (Issue 1)\\
Jan.30, 2025 & Yuanqi Xue & Second Draft\\
\bottomrule
\end{tabularx}
\end{table}

\section{Problem Statement}
\subsection{Problem}
Graph Neural Networks (GNNs) have shown strong performance in node classification, graph classification, and link prediction tasks. However, their black-box nature makes it difficult to understand their decision-making process, limiting their usage in critical areas such as medical diagnosis. To address this, the paper \href{https://ojs.aaai.org/index.php/AAAI/article/view/20898}{\textit{ProtGNN: Towards Self-Explaining Graph Neural Networks}} [1] proposes ProtGNN, a GNN model with built-in interpretability.
 

\subsection{Inputs and Outputs}
To maintain consistency with [1], we will be using the \href{https://pubmed.ncbi.nlm.nih.gov/1995902/}{\textit{MUTAG}} [2] dataset.\\

The input is a graph dataset. The output includes a trained model, classification results for the input graphs, and a set of learned prototypes, representing the key structural characteristics of each class.


\subsection{Stakeholders}
Researchers or students interested in the reproducibility and validation of [1].

\subsection{Environment}
For our implementation, we will use an NVIDIA GeForce RTX 3060 GPU for training and testing. However, testing should also be feasible on a personal laptop.


\section{Goals}
\begin{itemize}
    \item Implement the ProtGNN model as proposed in [1].
    \item Reproduce the classification accuracy on the MUTAG [2] dataset, and ensure that the identified prototypes (i.e., key structural characteristics of each class) have comparable quality to the original study.
\end{itemize}

\section{Stretch Goals}
Examine the reproducability of the paper [1] and validate its results.
\section{Challenge Level and Extras}
Challenge Level: Research Project \\
Extras: None

\section{Reference}
[1] Z. Zhang, Q. Liu, H. Wang, C. Lu, and C. Lee, “ProtGNN: Towards self-explaining graph neural networks,” in Proc. AAAI Conf. Artif. Intell., vol. 36, no. 8, pp. 9127–9135, Jun. 2022.\\

[2] A. K. Debnath, R. L. Lopez de Compadre, G. Debnath, A. J. Shusterman, and C. Hansch, “Structure-activity relationship of mutagenic aromatic and heteroaromatic nitro compounds. Correlation with molecular orbital energies and hydrophobicity,” J. Med. Chem., vol. 34, no. 2, pp. 786–797, Feb. 1991.

\end{document}
