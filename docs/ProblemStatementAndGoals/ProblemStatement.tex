\documentclass{article}

\usepackage{tabularx}
\usepackage{booktabs}

\title{Problem Statement and Goals}

\author{\authname}

\date{}

%% Comments

\usepackage{color}

\newif\ifcomments\commentstrue %displays comments
%\newif\ifcomments\commentsfalse %so that comments do not display

\ifcomments
\newcommand{\authornote}[3]{\textcolor{#1}{[#3 ---#2]}}
\newcommand{\todo}[1]{\textcolor{red}{[TODO: #1]}}
\else
\newcommand{\authornote}[3]{}
\newcommand{\todo}[1]{}
\fi

\newcommand{\wss}[1]{\authornote{blue}{SS}{#1}} 
\newcommand{\plt}[1]{\authornote{magenta}{TPLT}{#1}} %For explanation of the template
\newcommand{\an}[1]{\authornote{cyan}{Author}{#1}}

%% Common Parts

\newcommand{\progname}{ProgName} % PUT YOUR PROGRAM NAME HERE
\newcommand{\authname}{Team \#, Team Name
\\ Student 1 name
\\ Student 2 name
\\ Student 3 name
\\ Student 4 name} % AUTHOR NAMES                  

\usepackage{hyperref}
    \hypersetup{colorlinks=true, linkcolor=blue, citecolor=blue, filecolor=blue,
                urlcolor=blue, unicode=false}
    \urlstyle{same}
                                


\begin{document}

\maketitle

\begin{table}[hp]
\caption{Revision History} \label{TblRevisionHistory}
\begin{tabularx}{\textwidth}{llX}
\toprule
\textbf{Date} & \textbf{Developer(s)} & \textbf{Change}\\
\midrule
Jan.22, 2025 & Yuanqi Xue & First Draft\\
\bottomrule
\end{tabularx}
\end{table}

\section{Problem Statement}
\subsection{Problem}
Identifying frequent and structurally related subgraphs (i.e., network motifs) is computationally challenging due to the NP-hard nature of subgraph search and matching. Traditional methods are often inefficient and suffer from this complexity. The paper  \href{https://arxiv.org/abs/2206.01008}{\textit{Approximate Network Motif Mining via Graph Learning}} [1] introduces MotiFiesta, a machine learning-based approach that redefines motif mining as a node labeling task. Its fully differentiable framework enables efficient discovery of approximate motifs through graph representation learning.

\subsection{Inputs and Outputs}
The input is a synthetic graph dataset constructed according to the procedure described in the Appendix of [1]. The output is a trained model and the identified motifs (i.e., the frequent subgraphs) within the input graphs.


\subsection{Stakeholders}
Researchers or students interested in the reproducibility and validation of [1].

\subsection{Environment}
As stated in the original paper, training on the synthetic dataset was conducted using an NVIDIA GeForce GTX 1080 GPU, taking 4-8 hours, while decoding was performed on a personal laptop with a 1.6 GHz dual-core Intel Core i5 processor. For our implementation, we will use an NVIDIA GeForce RTX 3060 GPU for training and a personal laptop with a 2.80 GHz quad-core Intel Core i7 processor for decoding.


\section{Goals}
\begin{itemize}
    \item Create a synthetic dataset using the procedure described in the original paper.
    \item Implement the proposed method MotiFiesta.
    \item Reproduce the paper's results by identifying motifs from the synthetic dataset with comparable quality to those reported in the original study.
\end{itemize}

\section{Stretch Goals}
Examine the reproducability of the paper [1] and validate its results.
\section{Challenge Level and Extras}
Challenge Level: Research Project \\
Extras: None

\section{Reference}
[1] C. Oliver, D. Chen, V. Mallet, P. Philippopoulos, and K. Borgwardt, “Approximate network motif mining via graph learning,” arXiv:2206.01008, 2022.

\end{document}
